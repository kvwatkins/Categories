\section{Introduction}
\subsection{Abstract}
In the Typescript world we currently don't have a clear source  for how Typescript and its ecosystem surrounding functional programming relate to the categorical concepts. The current efforts in the community have been largely been comprised of  limited documentation surrounding functional programming  libraries and in partial blog post. This leads to a large barrier of entry for developers coming into the ecosystem either new or from another functional language. By Using a subset of Category Theory concepts as an abstract model we can create a generalized model that can be implemented into typescript.

For an abstract model to be useful it would have to be communicable across programming languages of different types meaning that concepts from the model would need to be recognizable in multiple languages modern functional languages. For the Typescript implementation to be practical inside typescripts type system it would need to be able to use existing typescript ecosystem constructs where possible, be low boilerplate, and performant in both the front and back-end ecosystems where it is used.

This document assumes the reader is a beginner with Haskell and Typescript. And presumes no mathematical knowledge outside of basic logic, algebra, and functions.

\subsection{Goals}
This document is broken into several parts \ldots \textbf{Part 1} seeks to first define an abstract model that a practical implementation can be based on. We choose to avoid typescript entirely in this section and turn to Haskell as the language of choice due to the existing documentation and libraries that define category theory concepts already. \textbf{Part 2} seeks to define an implementation and identify the caveats between the typescript representation and the Haskell representation of the model. \textbf{Part 3} is to be determined but its focus will be on communication between typescript and Haskell, translating domain types to the model, and migration schemes for existing representations of concepts. \textbf{Part 4} seeks to define existing functional architectures in terms of the abstract model as well as provide basic example implementations and possible improvements.