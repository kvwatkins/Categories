\subsection{Monoids}

%https://math.stackexchange.com/questions/1332690/monoid-as-a-single-object-category
Monoids are a mathematical concept found across different branches of mathematics.

%Set Theory
In set theory a monoid is a set equipped with a binary function that is associative and a unit element. An example is addition on the set of integers is equipped with the pseudo function $(+) :: a \rightarrow a \rightarrow a$.

Monoid $M$ is a set with a unit element $e$ and binary operation. Such that if $a,b,c \in M$ then \ldots

\begin{figure}[ht]
$$a \circ b \in M $$
$$(a \circ b) \circ c = a \circ (b \circ c)$$
$$e \circ a = a \circ e = a$$
\end{figure}

%Category Theory 
In category theory a monoid is a one object category with a set of morphisms that follow the rules of composition.

Given a single object category $C$. $C$'s object is $c$, $C$'s morphism $c \rightarrow c$. $C$'s unit is $1_c : c -> c$

\begin{figure}[ht]
$$f \circ g \in Morph(C)$$
$$(f \circ g) \circ h = f \circ (g \circ h) $$
$$1_c \circ f = f \circ 1_c = f$$
\end{figure}


Operations on $M$ are exactly the same as operations on $Morph(C)$ allowing us to regard the category $C$ as the monoid $M$. The correspondence is reversible given the category of $C = {c}$ we obtain a monoid $M$ whose elements are arrows of $C$ This allows us to extract a set from a single object category. 

because $c \in C$ and the hom-set of $c$ is $Hom_C(c,c)$ resulting in a Set monoid M whose elements are morphisms of $C$

In the Hask we can define a subcategory definition for the Monoid using a typeclass.
\begin{verbatim}
class Monoid m where
mempty  :: m
mappend :: m -> m -> m
\end{verbatim}