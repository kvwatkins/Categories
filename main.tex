\documentclass{article}

%Preamble
\title{Category Theory in Haskell}
\author{Kenneth Watkins}


\begin{document}

\maketitle

\section{What is a Category}
A Category consist of two things
\begin{itemize}
  \item Objects
  \item Arrows - Morphisms
\end{itemize}

To form a category objects and arrows must satisfy the following

\begin{itemize}
  \item Objects must have a an arrow originating from itself to itself as the Identity morphism. This arrow serves as a unit of composition such that when composed with any Arrow that either starts at A or ends at A it gives back the same arrow. So if f is an arrow f . (idA) = f
  \item Given an object A, B, C and two arrows, one from A to B and B to C there must exist a third arrow from A to C.
  \item Given 3 arrows f, g, h, the they must be associative h . (g . f) = (h . g) . f = h . g . f
\end{itemize}

\section{Haskell Types and the Category of Set}

A type is a set of values. As an example the Haskell type Bool is a two element set of True and False.

Sets can be finite or infinite

x :: Integer is saying x is an element of Integer

The category of sets is called Set. Its special because we can peak in at its objects.

In Set...
\begin{itemize}
  \item Objects are sets
  \item Arrows are morphisms
\end{itemize}

Intutions about Set
\begin{itemize}
  \item empty set has no elements
  \item there are special one element sets
  \item functions map elements of one set to elements of another set
  \item functions can map two elements to one but not one element to two
  \item there exist an identity function that maps each element of a set to itself

In Haskell the category of haskell types and functions is referred to as Hask. By forgetting the bottom hask is Set
\end{itemize}

\begin{center}
\begin{tabular}{||c c c c||} 
 \hline
 Set & Haskell Type & Typescript Type & Description \\ [0.5ex] 
 \hline\hline
 Empty Set & Void & never & Type not inhibited by any values \\
 \hline
Singleton Set & () "Unit" & void & Type has one value that always exist \\
 \hline
Two Element Set & Bool & boolean & true or false functions to this type are called predicates\\ [1ex] 
 \hline
\end{tabular}
\end{center}

\end{document}