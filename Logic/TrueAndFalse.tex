\subsection{Abstraction and Encoding}

Abstracting involves forgetting the things that are different between two things that have some shared similarity, post abstraction we are left only with the similarities between the things originally abstracted. A more elegant way 
of stating this is that ``we abstract over differences so that we may integrate by the similarities``. 

Encoding is the act of taking something in one form and transforming it into another form while preserving some essential essence of the original in the end resulting form. We do this in steps. The first step is abstracting from the starting form, leaving behind the similarity we wish to transform into the end form. In the second step we combine the simlarity with some thing that requires the similarity to produce the end form. Decoding is the same process except instead, we combine the similarity with something that requires it to produce the starting form.

\subsection{Examples of Encodings}

Informally the study of logic consist of encodings called propositions that when evaluated result in values that are either true or false. Simply put we encode some meaning by attaching symbols in a specific order which form words that in turn form sentences that can be evaluated to result in a value of true or false. As an example ``Combining the colors Yellow and Blue will result in Green`` which would evalute to true. When we read the sentence we are decoding them back to their original meaning. 

But what happens when the thing can't be evaluated? In logic we call it an absurdity. The classic example is the self referential proposition ``this sentence is false``.

Propositions have another encoding aside from sentences in natural language. As an example $Yellow + Blue = Green$ or $1 + 3 = 5$ would result in $true$ and $false$ respectfully.

\subsection{Logical Truth, the origin of truth}
What is truth? More specifically what is the meaning behind the encoding of true?

\subsection{Functions as Logical Encodings}
Logic can be encoded using just functions. A proposition that always returns true regardless of the interpretation is said to be a logical truth. $f(x,y)=x$